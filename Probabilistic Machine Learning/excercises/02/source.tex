\documentclass[UTF8]{article}
\usepackage{amsmath}
\usepackage{amsthm}
\title{Probability: Univariate Models}
\date{}
\begin{document}
\maketitle

\noindent 1.

\noindent a.

\begin{equation}
    P(H, e_1, e_2) = P(e_1, e_2 | H)P(H)\label{eq:1-1}
\end{equation}
 

\begin{align}
    P(H | e_1, e_2) &= \frac{P(H, e_1, e_2)}{P(e_1, e_2)}\notag\\
                    &= \frac{P(e_1, e_2 | H)P(H)}{P(e_1, e_2)}\label{eq:1-2}
\end{align}

Therefore, the second sets of numbers are sufficient for the calculation.

\noindent b.

from $E_1\perp E_2 | H$, we know that:

\begin{equation}
    P(e_1, e_2| H) = P(e_1|H)P(e_2|H)\label{eq:1-3}
\end{equation}

Therefore, from equations \ref{eq:1-3} and \ref{eq:1-2}, there are:

\begin{align}
    P(H | e_1, e_2) &= \frac{P(e_1, e_2 | H)P(H)}{P(e_1, e_2)}\notag\\
                    &= \frac{P(e_1|H)P(e_2|H)P(H)}{P(e_1, e_2)}
\end{align}

Therefore, the first sets of numbers are sufficient for the calculation.

\end{document}