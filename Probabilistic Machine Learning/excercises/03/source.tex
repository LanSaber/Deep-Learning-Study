\documentclass[UTF8]{article}
\usepackage{amsmath}
\usepackage{amsfonts}
\usepackage{amssymb}
\usepackage{amsthm}
\usepackage{mathrsfs}
\title{Probability: Multivariate Models}
\date{}
\begin{document}
\maketitle

\noindent 1.

\begin{align*}
    Cov(X, Y) &= \mathbb{E}[XY] - \mathbb{E}[X]\mathbb{E}[Y]\\
    &=\mathbb{E}[X^3] - \mathbb{E}[X]\mathbb{E}[X^2]
\end{align*}

Because $X\sim U(-1, 1)$, therefore $\mathbb{E}[X] = 0, \mathbb{E}[X^3] = 0$

\begin{align*}
    Cov(X, Y) &=\mathbb{E}[X^3] - \mathbb{E}[X]\mathbb{E}[X^2]\\
    &=0
\end{align*}

therefore

\begin{align*}
    \rho(X, Y) &=\frac{Cov(X, Y)}{\sqrt{\mathbb{V}[X]\mathbb{V}[Y]}}\\
    &=0
\end{align*}

Although $X$ and $Y$ are uncorrelated, it is definite that $Y$ is dependent on $X$.

\noindent 2.

To make the correlation coefficient meaningful, there must be $\mathbb{V}[X]>0$ and $\mathbb{V}[Y]>0$. Therefore, considering a random variable $Z = aX + Y$.

\begin{align*}
	\mathbb{V}[Z] &= \mathbb{V}[aX+Y]\\
	&=\mathbb{V}[X]a^2+2\text{Cov}[X, Y]a+\mathbb{V}[Y]\\
\end{align*}

Because $\mathbb{V}[Z]\ge 0$ for all $a$. Therefore, 

\begin{align*}
	\Delta &\le 0\\
	4\text{Cov}^2[X, Y] - 4\mathbb{V}[X]\mathbb{V}[Y]&\le 0\\
	\frac{\text{Cov}^2[X, Y]}{\mathbb{V}[X]\mathbb{V}[Y]}&\le 1\\
\end{align*}

So, there is $\rho^2\le 1$. Therefore, $-1\le\rho\le 1$.

\noindent 3.

\begin{align*}
	\text{Cov}[X, Y] &= \mathbb{E}[XY]-\mathbb{E}[X]\mathbb{E}[Y]\\
	&=\mathbb{E}[aX^2+bX] - \mathbb{E}[X]\mathbb{E}[aX+b]\\
	&=a\mathbb{E}[X^2] + b\mathbb{E}[X] - \mathbb{E}[X](a\mathbb{E}[X]+b)\\
	&= a\mathbb{V}[X]\\
\end{align*}

\begin{align*}
	\mathbb{V}[X]\mathbb{V}[Y] &= \mathbb{V}[X]\mathbb{V}[aX+b]\\
	&=a^2\mathbb{V}^2[X]
\end{align*}

Therefore, 
\begin{align*}
\rho(X, Y)&=\frac{	\text{Cov}[X, Y]}{\sqrt{\mathbb{V}[X]\mathbb{V}[Y]}}\\
&=\frac{a}{|a|}
\end{align*}

Therefore, if $a>0$, then $\rho(X, Y)=1$. If $a<0$, then $\rho(X, Y)=-1$.

\noindent 4.

\noindent a.

\begin{align*}
	\text{Cov}[Ax] &= \mathbb{E}[(Ax-\mathbb{E}[Ax])(Ax-\mathbb{E}[Ax])^T]\\
	&=\mathbb{E}[(Ax-A\mathbb{E}[x])(x^TA^T-\mathbb{E}^T[x]A^T)]\\
	&=\mathbb{E}[A(x-\mathbb{E}[x])(x^T-\mathbb{E}^T[x])A^T]\\
	&=A\mathbb{E}[(x-\mathbb{E}[x])(x-\mathbb{E}[x])^T]A^T\\
	&=A\Sigma A^T\\
\end{align*}

\noindent b.

If $C=AB$, then

\begin{align*}
	\text{tr}[AB] &= \Sigma_k c_{kk}\\
	&=\Sigma_k\Sigma_i a_{ki}b_{ik}\\
	& = \Sigma_k\Sigma_i b_{ik}a_{ki}\\
	& = \Sigma_i\Sigma_k b_{ik}a_{ki}\\
	& = \text{tr}[BA]
\end{align*}

\noindent c.

\begin{align*}
\mathbb{E}[x^TAx] &= \mathbb{E}[\text{tr}(x^TAx)]\\
&= \mathbb{E}[\text{tr}(Axx^T)]\\
&= \text{tr}(A\mathbb{E}[xx^T])\\
&= \text{tr}(A(\Sigma+mm^T))\\
&= \text{tr}(A\Sigma)+m^TAm\\
\end{align*}

\end{document}